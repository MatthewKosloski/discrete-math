\section*{1.5 Quantifiers}
\addcontentsline{toc}{section}{1.5 Quantifiers}

\subsection*{Propositional Function}
\addcontentsline{toc}{subsection}{Propositional Function}

Consider the statement
\begin{center}
\textit{p}: $n$ is an odd integer.
\end{center}

It is not a proposition because the truth value is predicated on $n$.  Most statements in math are similar to this, therefore, we must extend the system of logic to include such statements.\\

A \textbf{propositional function} $P(x)$ is a function with respect to a set $D$, where each $x \in D$.  We call $D$ the \textbf{domain of discourse}.

\subsubsection*{Example 1}

Let
\begin{align*}
P(n): n \text{ is an odd integer.}
\end{align*}

Then $P$ is a propositional function with respect to set $Z^+$ (set of positive integers).  For each $n \in Z^+$, $P(n)$ is a proposition.\\

Let
\begin{align*}
P(1)&: 1 \text{ is an odd integer},\\
P(2)&: 2 \text{ is an odd integer}
\end{align*}

Clearly, $P(1)$ is \textit{true} and $P(2)$ is \textit{false}.\\

\textbf{Remember: A propositional function $P$ by itself is neither \textit{true} nor \textit{false}; however, for each $x$ in the domain of discourse, $P(x)$ is a proposition and is, therefore, either \textit{true} or \textit{false}.}\\

The following is a list of valid propositional functions:

\begin{enumerate}[label=(\alph*)]
\item $n^2 + 2n$ is an odd integer (domain of discourse = $Z^+$)
\item $x^2 - x - 6 = 0$ (domain of discourse = $R$)
\item The restaurant rated over two stars (domain of discourse = rated restaurants)
\end{enumerate}

(c) by itself is not a proposition; however, ``restaurant'' can be replaced with a restaurant, such as ``Portillo's'', to produce a proposition.

\subsection*{Universally Quantified Statement}
\addcontentsline{toc}{subsection}{Universally Quantified Statement}

Let $P$ be a propositional function with domain of discourse $D$.  Then,

\[
    \forall x P(x)
\]

is said to be a \textbf{universally quantified statement}.  It is \textit{true} if $P(x)$ is \textit{true} for every $x$ in $D$.  It is \textit{false} if $P(x)$ is \textit{false} for at least one $x$ in $D$.  Such an $x$ that makes $P(x)$ \textit{false} is a \textbf{counterexample}.  The symbol $\forall$ may be read ``for every'', ``for all'', or ``for any.''\\

To prove that

\[
    \forall x P(x)
\]

is \textit{true}, we must examine \textit{every} value of $x$ in set $D$ to show that for every $x$, $P(x)$ is \textit{true}.  However, it is much easier to find a counterexample such that $P(x)$ is \textit{false}.\\

\textbf{Remember: To disprove $\forall x P(x)$, find one $x$ in set $D$ such that $P(x)$ is \textit{false}.}

\subsubsection*{Example 2}

Consider the universally quantified statement with domain of discourse \textbf{R}.

\[
    \forall x (x^2 \geq 0).
\]

The statement is \textit{true} because, \textit{for every} real number $x$, $x^2 \geq 0$.

\subsubsection*{Example 3}

The universally quantified statement
\begin{center}
    for every real number $x$, if $x > 1$, then $x + 1 > 1$
\end{center}

is \textit{true}.  If $x \leq 1$, the hypothesis $x > 1$ is \textit{false}, therefore, the proposition is \textit{true}.  If $x > 1$, since

\[
    x + 1 > x \text{ and } x > 1,    
\]

we conclude that $x + 1 > 1$, so the conclusion is \textit{true}.  If $x > 1$, the hypothesis and conclusion are both \textit{true}, hence the universally quantified statement is \textit{true}.

\subsection*{Existentially Quantified Statement}
\addcontentsline{toc}{subsection}{Existentially Quantified Statement}

Let $P$ be a propositional function with domain of discourse $D$.  Then,

\[
    \exists x P(x)
\]

is said to be an \textbf{existentially quantified statement}.  It is \textit{true} if $P(x)$ is \textit{true} for at least one $x$ in $D$.  It is \textit{false} if $P(x)$ is \textit{false} for every $x$ in $D$.  The symbol $\exists$ is read as ``there exists'', ``for some'', or ``for at least one.''

\subsubsection*{Example 4}

Consider the existentially quantified statement with domain of discourse \textbf{R}.

\[
    \exists x \left( \frac{x}{x^2 + 1} = \frac{2}{5} \right)
\]

If we can find one real number $x$ such that $\left(\frac{x}{x^2 + 1} = \frac{2}{5}\right)$, then the existentially quantified statement is \textit{true}.\\

If $x = 2$, then,

\[
    \left( \frac{2}{2^2 + 1} = \frac{2}{5} \right)
\]

therefore, $\exists x \left( \frac{x}{x^2 + 1} = \frac{2}{5} \right)$ is \textit{true}.

\subsubsection*{Example 5}

Verify that the following existentially quantified statement is \textit{false}.

\[
    \exists x \in \textbf{R} \left( \frac{1}{x^2 + 1} > 1 \right)
\]

It is \textit{false} if $\left( \frac{1}{x^2 + 1} > 1 \right)$ is \textit{false} for every real number $x$.

\[
    \exists x \in \textbf{R} \left( \frac{1}{x^2 + 1} > 1 \right)
\]

is \textit{false} when

\[
    \forall x \in \textbf{R} \left( \frac{1}{x^2 + 1} \leq 1 \right)
\]

is \textit{true}.  Therefore, we must show that $\left( \frac{1}{x^2 + 1} \leq 1 \right)$ is \textit{true} for every real number $x$.

\subsection*{De Morgan's Laws for Logic}
\addcontentsline{toc}{subsection}{De Morgan's Laws for Logic}

\begin{align*}
    \lnot(\forall x P(x)) \equiv \exists x \lnot P(x)\\
    \lnot(\exists x P(x)) \equiv \forall x \lnot P(x)\\
\end{align*}

\subsubsection{Example 6}

Write the following statement symbolically and write the negation symbolically and in words.
\begin{center}
    Every rock fan loves U2.
\end{center}

Let
\begin{align*}
    P(x) &= x \text{ loves U2},\\
    D &= \text{ the set of rock fans}.
\end{align*}

Then, the statement is symbolically written as 

\[
    \forall x P(x).
\]

And the negation, $\lnot (\forall x P(x))$, is

\[
    \exists x \lnot P(x),
\]

which can be read as

\begin{center}
    There exists a rock fan who does not love U2.
\end{center}

\subsubsection{Example 7}

Write the following statement symbolically and write the negation symbolically and in words.
\begin{center}
    Some birds cannot fly
\end{center}

Let
\begin{align*}
    P(x) &= x \text{ can fly},\\
    D &= \text{ the set of all birds}.
\end{align*}

Then, the statement is symbolically written as 

\[
    \exists x \lnot P(x).
\]

And the negation, $\lnot (\exists x \lnot P(x))$, is

\[
    \forall x \lnot\lnot P(x) = \forall x P(x)
\]

which can be read as

\begin{center}
    All birds can fly.
\end{center}

\subsection*{Generalizing Propositions}
\addcontentsline{toc}{subsection}{Generalizing Propositions}

A universally quantified statement such as the one above generalizes the proposition

\[
    p_1 \land p_2 \land \dots \land p_n
\]

in the sense that the individual propositions are replaced by a family $P(x)$.  In other words,

\[
    p_1 \land p_2 \land \dots \land p_n \equiv \forall x P(x).
\]

Similarly, an existentially quantified statement generalizes the proposition

\[
    p_1 \vee p_2 \vee \dots \vee p_n
\]

in the sense that the individual propositions are replaced by a family $P(x)$.  In other words,

\[
    p_1 \vee p_2 \vee \dots \vee p_n \equiv \exists x P(x).
\]

\subsubsection{Example 7}

If the domain of discourse of the propositional function \textit{P} is \{-1, 0, 1\}, then $\forall x P(x)$ is equivalent to

\[
    P(-1) \land P(0) \land P(1).
\]

And by De Morgan's laws of logic, the negation $\lnot(P(-1) \land P(0) \land P(1))$ is logically equivalent to

\[
    \lnot P(-1) \vee \lnot P(0) \vee \lnot P(1) \equiv \exists x \lnot P(x)
\]

\subsubsection{Example 8}

The following statement

\begin{center}
    All that glitters is not gold
\end{center}

has multiple interpretations.  Such interpretations include

\begin{enumerate}[label=(\alph*)]
\item Every object that glitters is not gold
\item Some object that glitters is not gold
\end{enumerate}

If we let
\begin{align*}
    P(x)&: x \text{ glitters,}\\
    Q(x)&: x \text{ is gold,}
\end{align*}

then (a) is symbolically written as

\[
    \forall x (P(x) \rightarrow \lnot Q(x))
\]

and (b) is symbolically written as

\[
    \exists x (P(x) \land \lnot Q(x)).
\]

Since $\lnot(p \rightarrow q) \equiv p \land \lnot q$, (b) is logically equivalent to

\[
    \exists x \lnot (P(x) \rightarrow Q(x)).
\]

Since  $\exists x \lnot P(x) \equiv \lnot(\forall x P(x))$, (b) is also logically equivalent to

\[
    \lnot (\forall x P(x) \rightarrow Q(x)).
\]

Thus, the correct interpretation results from negating the original statement.

\clearpage

\subsection*{Rules of Inference for Quantified Statements}
\addcontentsline{toc}{subsection}{Rules of Inference for Quantified Statements}

\begin{table}[h]
\centering
\begin{tabular}{ll}
  
    \textbf{Universal instantiation}&
    \begin{tabular}{l@{\,}l@{\,}l@{\,}}
    $\forall x P(x)$ & & \\
    \hline
    $\therefore P(d) \text{ if \textit{d}} \in D$ & &
    \end{tabular}\\& \\ 
    
    \textbf{Universal generalization}&
    \begin{tabular}{l@{\,}l@{\,}l@{\,}}
    $P(d) \text{ for every \textit{d}} \in D$ & & \\
    \hline
    $\therefore \forall x P(x)$ & &
    \end{tabular}\\& \\
    
    \textbf{Existential instantiation}&
    \begin{tabular}{l@{\,}l@{\,}l@{\,}}
    $\exists x P(x)$ & & \\
    \hline
    $\therefore P(d) \text{ for some \textit{d}} \in D$ & &
    \end{tabular}\\& \\ 
    
    \textbf{Existential generalization}&
    \begin{tabular}{l@{\,}l@{\,}l@{\,}}
    $P(d) \text{ for some \textit{d}} \in D$ & & \\
    \hline
    $\therefore \exists x P(x)$ & &
    \end{tabular}\\& \\
  
\end{tabular}
\end{table}

\subsubsection{Example 9}

Let
\begin{align*}
P(x)&: x \text{ owns a laptop}\\
D&: \{x \mid x \text{ is a student in MA 309}\}
\end{align*}

Suppose that Matthew, who is taking MA 309, owns a laptop; in symbols, $P(Matthew)$ is \textit{true}.  Then, by existential generalization, $\exists x P(x)$ is \textit{true}.

\subsubsection{Example 10}

Write the argument symbolically and then, using rules of inference, show that it is valid.
\begin{center}
    For every real number $x$, if $x$ is an integer, then $x$ is a rational number.  The number $\sqrt{2}$ is not rational.  Therefore, $\sqrt{2}$ is not an integer.
\end{center}

\clearpage

Let
\begin{align*}
P(x)&: x \text{ is an integer,}\\
Q(x)&: x \text{ is rational.}
\end{align*}

Then the argument is written symbolically as 

\begin{table}[h]
\centering
\begin{tabular}{l@{\,}l@{\,}l@{\,}}
  $\forall x \in \textbf{R}(P(x) \rightarrow Q(x))$ & & \\
  $\lnot Q(\sqrt{2})$ & & \\
\hline
 $\therefore \lnot P(\sqrt{2}) $ & &
\end{tabular}
\end{table}

Since $\sqrt{2} \in \textbf{R}$, we may use Universal instantiation to conclude $P(\sqrt{2}) \rightarrow Q(\pi)$.  Combining $P(\sqrt{2}) \rightarrow Q(\pi)$ and $\lnot Q(\pi)$, we use modus tollens to conclude $\lnot P(\sqrt{2})$.






