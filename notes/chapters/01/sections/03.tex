\section*{1.3 Conditional Propositions and Logical Equivalence}
\addcontentsline{toc}{section}{1.3 Conditional Propositions and Logical Equivalence}

\subsection*{Conditional Proposition}
\addcontentsline{toc}{subsection}{Conditional Proposition}

Consider the following proposition:

\begin{center}
If it is raining outside, then I will bring an umbrella.
\end{center}

The above proposition is called a \textbf{conditional proposition}, and it states that on the condition that it is raining outside, then I will bring an umbrella.\\

If we let
\begin{align*}
p&: \text{It is raining outside,}\\
q&: \text{I will bring an umbrella,}
\end{align*}

we can denote the conditional proposition as

\[
    p \rightarrow q.
\]

The above can be pronounced as ``if \textit{p} then \textit{q}'' or ``\textit{p} implies \textit{q}.'' The proposition \textit{p} is called the \textbf{hypothesis} or \textbf{sufficient condition}, and the proposition \textit{q} is called the \textbf{conclusion} or \textbf{necessary condition}.\\

How do you determine the truth value of a conditional proposition, such as the one above?  Suppose I say,

\begin{center}

If I buy a car, then I will let you drive it.

\end{center}

If I end up buying a car and letting you drive it, then the statement is \textit{true}.  However, if I do buy the car and do \textit{not} let you drive it, then the statement is \textit{false}.  If I do \textit{not} buy a car, the statement is still true (there is no car for you to drive, but there may be one in the future).\\  

The following table illustrates the truth value of $p \rightarrow q$:

\begin{table}[h]
\centering
\begin{tabular}{|c|c|c|}
\hline
\textit{p} & \textit{q} & \textit{p} $\rightarrow$ \textit{q} \\ \hline
T          & T          & T       \\ \hline
T          & F          & F       \\ \hline
F          & T          & T       \\ \hline
F          & F          & T       \\ \hline
\end{tabular}
\end{table}

From this, it is clear that a conditional proposition is only \textit{false} when the hypothesis is \textit{true} and the conclusion is \textit{false}.\\  

\subsection*{True by Default}
\addcontentsline{toc}{subsection}{True by Default}

To justify how a conditional proposition is always \textit{true} when \textit{p} is \textit{false}, consider the following proposition:
\begin{center}
    For all real numbers \textit{x}, if $x > 0$, then $x^2 > 0$
\end{center}

If we let
\begin{align*}
    P(x)&: x > 0,\\
    Q(x)&: x^2 > 0
\end{align*}

Then we can denote the proposition as
\begin{center}
    if \textit{P(x)} then \textit{Q(x)}.
\end{center}

If we let $x = -2$, then \textit{P(-2)} is \textit{false} and \textit{Q(-2)} is \textit{true}.  If we let $x = 0$, then \textit{P(0)} and \textit{Q(0)} are both \textit{false}.  This is why we must define $p \rightarrow q$ to be \textit{true} no matter what the truth value of \textit{p} is.  This is called \textbf{true by default}.

\subsection*{Operator Precedence}
\addcontentsline{toc}{subsection}{Operator Precedence}

In conditional propositions that involve logical operators $\land, \vee, \lnot, and \rightarrow$, the conditional operator $\rightarrow$ is evaluated last.  Therefore, we now have the following order of precedence:

\begin{table}[h]
\centering
\begin{tabular}{|c|c|}
\hline
\textit{Operator} & \textit{Precedence} \\ \hline
$\lnot$           & 1                   \\ \hline
$\land$           & 2                   \\ \hline
$\vee$            & 3                   \\ \hline
$\rightarrow$     & 4                   \\ \hline
\end{tabular}
\end{table}

Let \textit{p} be \textit{true}, \textit{q} be \textit{false}, and \textit{r} be \textit{true}.  Evaluate

\begin{enumerate}[label=(\alph*)]
\item $p \land q \rightarrow r$
\item $p \vee q \rightarrow \lnot r$
\item $p \land (q \rightarrow r)$
\item $p \rightarrow (q \rightarrow r)$
\end{enumerate}

\begin{enumerate}[label=(\alph*)]
\item We first evaluate $p \land q$, which is $false$, and then we evaluate $p \land q \rightarrow r$, which is \textit{true}.
\item We first evaluate $\lnot r$, which is $false$, then we evaluate $p \vee q$, which is $true$, and finally we evalute the entire proposition $p \vee q \rightarrow \lnot r$, which is $false$.
\item We first evaluate $(q \rightarrow r)$, which is $true$, and then evaluate $p \land (q \rightarrow r)$, which is $true$.
\item We first evaluate $(q \rightarrow r)$, which is $true$, and then we evaluate $p \rightarrow (q \rightarrow r)$, which is $true$.
\end{enumerate}

\subsection*{Rewriting Propositions as Conditional Propositions}
\addcontentsline{toc}{subsection}{Rewriting Propositions as Conditional Propositions}

For each proposition, rewrite it as a conditional proposition in the form $p \rightarrow q$:

\begin{enumerate}[label=(\alph*)]
\item Mary will be a good student if she studies hard.
\item John takes calculus only if he has sophomore, junior, or senior standing.
\item When you sing, my ears hurt.
\item A necessary condition for the Cubs to win the World Series is that they sign a right-handed relief pitcher.
\item A sufficient condition for Maria to visit France is that she goes to the Eiffel Tower.
\end{enumerate}

(a) - (e) can be rewritten as

\begin{enumerate}[label=(\alph*)]
\item If Mary studies hard, then she will be a good student.
\item ``\textit{p} only if \textit{q}'' is the same as ``if \textit{p} then \textit{q}'', therefore, the proposition can be rewritten as:
\begin{center}
If John takes calculus, then he has sophomore, junior, or senior standing.
\end{center}
\item \textit{When} is the same as \textit{if}; thus the proposition is rewritten as
\begin{center}
If you sing, then my ears hurt.
\end{center}
\item A \textbf{necessary condition} is a condition that is necessary for an outcome but does not guarantee the outcome, therefore, we can rewrite it as
\begin{center}
If the Cubs win the World Series, then they signed a right-handed relief pitcher.
\end{center}
\item A \textbf{sufficient condition} is a condition that, when met, guarantees an outcome; however, if it is \textit{not} met, the outcome is still possible.  We can rewrite this proposition as 
\begin{center}
If Maria goes to the Eiffel tower, then she visits France.
\end{center}
\end{enumerate}

\subsection*{Converse}
\addcontentsline{toc}{subsection}{Converse}

The \textbf{converse} of $p \rightarrow q$ is $q \rightarrow p$.  

Let
\begin{align*}
p&: 1 > 2,\\
q&: 4 < 8.
\end{align*}

Since \textit{p} is false and \textit{q} is true, $p \rightarrow q$ is \textit{true}.  However, its converse, $q \rightarrow p$ is \textit{false}.  Thus, a conditional proposition can be \textit{true} while its converse is \textit{false}.\\

For example, if we have the following conditional proposition $p \rightarrow q$, we can write its converse symbolically and in words.
\begin{center}
If Jerry receives a scholarship, then he will go to college.
\end{center}

Let
\begin{align*}
p&: \text{Jerry receives a scholarship},\\
q&: \text{Jerry goes to college}.
\end{align*}

The converse is symbolically expressed as $q \rightarrow p$, which can be written in words as

\begin{center}
If Jerry goes to college, then he receives a scholarship.
\end{center}

Now, if Jerry does \textit{not} receive a scholarship but goes to college anyway, find the truth values of (a) and (b).
\begin{enumerate}[label=(\alph*)]
\item $p \rightarrow q$
\item $q \rightarrow p$
\end{enumerate}

\begin{enumerate}[label=(\alph*)]
\item Since Jerry did \textit{not} receive a scholarship, \textit{p} is \textit{false}, but he is still going to college, so \textit{q} is \textit{true}.  Therefore, $p \rightarrow q$ is \textit{true}.
\item Since Jerry is going to college, \textit{q} is \textit{true}.  However, he did \textit{not} receive a scholarship, so \textit{p} is \textit{false}.  Therefore, $q \rightarrow p$ is \textit{false}.
\end{enumerate}

\subsection*{Biconditional Proposition}
\addcontentsline{toc}{subsection}{Biconditional Proposition}

A \textbf{biconditional proposition}, expressed as $p \leftrightarrow q$ or ``\textit{p} if and only if \textit{q}'', is true when \textit{p} and \textit{q} have the same truth values.  Thus, the truth value of $p \leftrightarrow q$ is defined by the following truth table:

\begin{table}[h]
\centering
\begin{tabular}{|c|c|c|}
\hline
\textit{p} & \textit{q} & $p \leftrightarrow q$ \\ \hline
T          & T          & T       \\ \hline
T          & F          & F       \\ \hline
F          & T          & F       \\ \hline
F          & F          & T       \\ \hline
\end{tabular}
\end{table}

In a biconditional proposition, \textit{p} is both necessary and sufficient for \textit{q}.

\subsection*{Logical Equivalence}
\addcontentsline{toc}{subsection}{Logical Equivalence}

Propositions are said to be \textbf{logically equivalent} if they have the same truth value, regardless of the truth values of their constituent propositions $p_1, \dots, p_n$.  If \textit{P} and \textit{Q} are made up of the propositions $p_1, \dots, p_n$, we say \textit{P} and \textit{Q} are logically equivalent and write
\[
    P \equiv Q
\]

provided that, given any truth values of $p_1, \dots, p_n$, \textit{P} and \textit{Q} are both \textit{true} or \textit{false}.\\

For example, we can show that the negation of $p \rightarrow q$ is logically equivalent to $p \land \lnot q$.  That is, 
\[
    \lnot (p \rightarrow q) \equiv p \land \lnot q.
\]

\begin{table}[h]
\centering
\begin{tabular}{|c|c|c|c|c|c|}
\hline
\textit{p} & \textit{q} & $p \rightarrow q$ & $\lnot (p \rightarrow q)$ & $\lnot q$ & $p \land \lnot q$ \\ \hline
T & T & T & F & F & F \\ \hline
T & F & F & T & T & T \\ \hline
F & T & T & F & F & F \\ \hline
F & F & T & F & T & F \\ \hline
\end{tabular}
\end{table}

The above demonstrates that $\lnot (p \rightarrow q)$ is logically equivalent to $p \land \lnot q$ regardless of the truth values of \textit{p} and \textit{q}.\\

We can use the logical equivalence of $\lnot (p \rightarrow q)$ and $p \land \lnot q$ to help us write the negation of conditional propositions.  For example, to negate
\begin{center}
    if Jerry receives a scholarship, then he goes to college,
\end{center}

we let
\begin{align*}
p&: \text{Jerry receives a scholarship},\\
q&: \text{Jerry goes to college.}
\end{align*}

The above proposition can be symbolically written as $p \rightarrow q$, and its negation is $\lnot (p \rightarrow q)$.  Since $\lnot (p \rightarrow q)$ is logically equivalent to $p \land \lnot q$, we can negate the proposition by expressing $p \land \lnot q$ as words like so:
\begin{center}
    Jerry receives a scholarship and he does not go to college.
\end{center}

\textbf{Remember: When evaluating conditional propositions, it is easier to work with the logical operators $\land$, $\vee$, and $\lnot$ than the conditional operator $\rightarrow$.}\\

Additionally, $p \leftrightarrow q \equiv (p \rightarrow q) \land (q \rightarrow p)$.  This is demonstrated by the following truth table:

\begin{table}[h]
\centering
\begin{tabular}{|c|c|c|c|c|c}
\hline
\textit{p} & \textit{q} & $p \leftrightarrow q$ & $p \rightarrow q$ & $q \rightarrow p$ & $(p \rightarrow q) \land (q \rightarrow p)$  \\ \hline
T & T & T & T & T & T \\ \hline
T & F & F & F & T & T \\ \hline
F & T & F & T & F & F \\ \hline
F & F & T & T & T & T \\ \hline
\end{tabular}
\end{table}

\subsection*{De Morgan's Laws for Logic}
\addcontentsline{toc}{subsection}{De Morgan's Laws for Logic}

De Morgan has the following two laws for logic:

\begin{align*}
    \lnot (p \vee q) &\equiv \lnot p \land \lnot q,\\
    \lnot (p \land q) &\equiv \lnot p \vee \lnot q
\end{align*}

\clearpage

For the first law, we can demonstrate that $\lnot (p \vee q)$ is logically equivalent to $\lnot p \land \lnot q$ using the following truth table:

\begin{table}[h]
\centering
\begin{tabular}{|c|c|c|c|c|c|c|}
\hline
\textit{p} & \textit{q} & $p \vee q$ & $\lnot (p \vee q)$ & $\lnot p$ & $\lnot q$ & $\lnot p \land \lnot q$ \\ \hline
T & T & T & F & F & F & F \\ \hline
T & F & T & F & F & T & F \\ \hline
F & T & T & F & T & F & F \\ \hline
F & F & F & T & T & T & T \\ \hline
\end{tabular}
\end{table}

Additionally, For the second law, we can demonstrate that $\lnot (p \land q)$ is logically equivalent to $\lnot p \vee \lnot q$ using the following truth table:

\begin{table}[h]
\centering
\begin{tabular}{|c|c|c|c|c|c|c|}
\hline
\textit{p} & \textit{q} & $p \land q$ & $\lnot (p \land q)$ & $\lnot p$ & $\lnot q$ & $\lnot p \vee \lnot q$ \\ \hline
T & T & T & F & F & F & F \\ \hline
T & F & F & T & F & T & T \\ \hline
F & T & F & T & T & F & T \\ \hline
F & F & F & T & T & T & T \\ \hline
\end{tabular}
\end{table}

\subsection*{Contrapositive}
\addcontentsline{toc}{subsection}{Contrapositive}

The \textbf{contrapositive} of the conditional proposition $p \rightarrow q$ is the proposition $\lnot q \rightarrow \lnot p$.\\

For example, consider the following proposition (Assume the network is \textit{not} down and Dale can access the Internet):

\begin{center}
    If the network is down, then Dale cannot access the Internet
\end{center}

Let
\begin{align*}
    p&: \text{The network is down,}\\
    q&: \text{Dale cannot access the Internet}
\end{align*}

The given proposition written symbolically is
\begin{center}
    $p \rightarrow q.$
\end{center}

Since the network is \textit{not} down, the hypothesis \textit{p} is \textit{false}, therefore, the proposition is \textit{true}.  The contrapositive can be written symbolically as
\begin{center}
    $\lnot q \rightarrow \lnot p$
\end{center}

and in words
\begin{center}
    If Dale can access the Internet, then the network is not down.
\end{center}

Since the hypothesis $\lnot q$ is \textit{true} and the conclusion $\lnot p$ is \textit{true}, the contrapositive is \textit{true}.\\

Thus, a conditional proposition and its contrapositive are logically equivalent, as demonstrated in this truth table:

\begin{table}[h]
\centering
\begin{tabular}{|c|c|c|c|}
\hline
\textit{p} & \textit{q} & $p \rightarrow q$ & $\lnot q \rightarrow \lnot p$ \\ \hline
T & T & T & T \\ \hline
T & F & F & F \\ \hline
F & T & T & T \\ \hline
F & F & T & T \\ \hline
\end{tabular}
\end{table}


























