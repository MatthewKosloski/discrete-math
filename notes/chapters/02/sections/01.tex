\section*{2.1 Mathematical Systems, Direct Proofs, and Counterexamples}
\addcontentsline{toc}{section}{2.1 Mathematical Systems, Direct Proofs, and Counterexamples}

A \textbf{mathematical system} consists of \textbf{axioms}, which are assumed to be true and \textbf{definitions}, which are used to create new concepts in terms of existing ones.  Within a mathematical system, we can derive a \textbf{theorem}, which is a proposition that has been proved to be true.\\

There are two special types of theorems:

\begin{itemize}
    \item \textbf{lemma:} not interesting in its own right, but useful for proving other theorems.
    \item \textbf{corollary:} follows easily from another theorem.
\end{itemize}

A \textbf{proof} is an argument that establishes the truth of a theorem.  Logic is used to analyze proofs.

\subsection*{Mathematical Systems}
\addcontentsline{toc}{subsection}{Mathematical Systems}

\subsubsection*{Example 1}

One example of a mathematical system is Euclidean geometry.  This mathematical system contains the following theorem and corollary:

\begin{itemize}
    \item \textbf{theorem:} If two sides of a triangle are equal, then the angles opposite them are equal.
    \item \textbf{corollary:} If a triangle is equilateral, then it is equiangular.
\end{itemize}

The corollary follows immediately from the theorem.

\subsubsection*{Example 2}

Examples of theorems about real numbers are:

\begin{itemize}
    \item $x \times 0 = 0$ for every real number $x$.
    \item For all real numbers $x, y, z$, if $x \leq y$ and $y \leq z$, then $x \leq z$.
\end{itemize}

\subsection*{Direct Proofs}
\addcontentsline{toc}{subsection}{Direct Proofs}

Theorems are often of the form
\begin{center}
    For all $x_1, x_2, \dots, x_n$, if $p(x_1, x_2, \dots, x_n)$, then $q(x_1, x_2, \dots, x_n)$.
\end{center}

The above is true provided that the conditional proposition $p \rightarrow q$ is true for all $x_1, x_2, \dots, x_n$ in the domain.  A \textbf{direct proof} assumes that $p(x_1, x_2, \dots, x_n)$ is true and then, using $p(x_1, x_2, \dots, x_n)$ as well as other axioms, definitions, previously derived theorems, and rules of inference, shows directly that $q(x_1, x_2, \dots, x_n)$ is true.\\

For example, to use the terms ``even integer'' and ``odd integer'' in a proof, we must first define them.
\begin{center}
    An integer $n$ is \textit{even} if there exists an integer \textit{k} such that $n = 2k$.\\
    An integer $n$ is \textit{odd} if there exists an integer \textit{k} such that $n = 2k + 1$.
\end{center}

\subsubsection*{Example 3}

Give a direct proof of the following statement:
\begin{center}
    For all integers \textit{m} and \textit{n}, if \textit{m} is odd and \textit{n} is even, then $m + n$ is odd.
\end{center}

We begin by first writing out the hypothesis and conclusion:

\begin{table}[h]
\centering
\begin{tabular}{l}
\textit{m} is odd and \textit{n} is even. (Hypothesis)\\
$\dots$\\
$m + n$ is odd. (Conclusion)
\end{tabular}
\end{table}

\clearpage

We begin to fill in the gaps using the definitions of even and odd:

\begin{table}[h]
\centering
\begin{tabular}{l}
\textit{m} is odd and \textit{n} is even. (Hypothesis)\\
\textit{m} = $2k_1 + 1$ ($k_1$ is an integer)\\
\textit{n} = $2k_2$ ($k_2$ is an integer)\\
$\dots$\\
$m + n$ is odd. (Conclusion)
\end{tabular}
\end{table}

Finally, to derive our conclusion, we use the definition of odd once again:
\begin{align*}
    m + n &= (2k_1 + 1) + 2k_2\\
    &= 2(k_1 + k_2) + 1
\end{align*}

Our final proof:

\begin{table}[h]
\centering
\begin{tabular}{l}
\textit{m} is odd and \textit{n} is even. (Hypothesis)\\
\textit{m} = $2k_1 + 1$ ($k_1$ is an integer)\\
\textit{n} = $2k_2$ ($k_2$ is an integer)\\
$m + n = 2(k_1 + k_2) + 1$\\
$m + n$ is odd. (Conclusion)
\end{tabular}
\end{table}

\subsubsection*{Example 4}

Give a direct proof of the following statement:
\begin{center}
    For all sets $X$, $Y$, and $Z$, $X \intersection (Y - Z) = (X \intersection Y) - (X \intersection Z)$
\end{center}

We begin by first writing out the hypothesis and conclusion:

\begin{table}[h]
\centering
\begin{tabular}{l}
$X$, $Y$, and $Z$ are sets. (Hypothesis)\\
$\dots$\\
$X \intersection (Y - Z) = (X \intersection Y) - (X \intersection Z)$ (Conclusion)
\end{tabular}
\end{table}

We are trying to conclude that sets $X \intersection (Y - Z)$ and $(X \intersection Y) - (X \intersection Z)$ are equal.  Recall the definition of set equality:
\begin{center}
    For every \textit{x}, if $x \in X$, then $x \in Y$\\
    For every \textit{x}, if $x \in Y$, then $x \in X$
\end{center}

We begin to fill in the gaps using the definition of set equality:

\begin{table}[h]
\centering
\begin{tabular}{l}
$X$, $Y$, and $Z$ are sets. (Hypothesis)\\
If $x \in X \intersection (Y - Z)$, then $x \in (X \intersection Y) - (X \intersection Z)$\\
If $x \in (X \intersection Y) - (X \intersection Z)$, then $x \in X \intersection (Y - Z)$\\
$\dots$\\
$X \intersection (Y - Z) = (X \intersection Y) - (X \intersection Z)$ (Conclusion)
\end{tabular}
\end{table}

To derive our conclusion, we must first prove
\begin{center}
    If $x \in X \intersection (Y - Z)$, then $x \in (X \intersection Y) - (X \intersection Z)$
\end{center}

and 
\begin{center}
    If $x \in (X \intersection Y) - (X \intersection Z)$, then $x \in X \intersection (Y - Z)$.
\end{center}

To prove the former, let $x \in X \intersection (Y - Z)$.  By the definition of intersection, if $x \in X \intersection (Y - Z)$, then $x \in X$ and $x \in Y - Z$.  By the definition of set difference, if $x \in Y - Z$, then $x \in Y$, but $x \notin Z$.  If $x \in X$ and $x \in Y$, then $x \in X \intersection Y$.  If $x \in X \intersection Y$ and $x \notin Z$, then $x \notin X \intersection Z$.  Therefore, if $x \in X \intersection (Y - Z)$, then $x \in (X \intersection Y) - (X \intersection Z)$.\\

To prove the latter, let $x \in (X \intersection Y) - (X \intersection Z)$.  By set difference, if $x \in (X \intersection Y) - (X \intersection Z)$, then $x \in (X \intersection Y)$ and $x \notin (X \intersection Z)$.  If $x \in (X \intersection Y)$, then, by intersection, $x \in X$ and $x \in Y$.  If $x \in X$ and $x \notin X \intersection Z$, then $X \notin Z$.  By the definition of set difference, since $x \in Y$ and $x \notin Z$, $x \in Y - Z$.  Finally, since $x \in X$ and $x \in Y - Z$, then $x \in X \intersection (Y - Z)$.  Therefore, if $x \in (X \intersection Y) - (X \intersection Z)$, then $x \in X \intersection (Y - Z)$.\\

Since we proved both of these equations, it follows that

\begin{center}
    $X \intersection (Y - Z) = (X \intersection Y) - (X \intersection Z)$.
\end{center}

The following example illustrates the use of \textbf{subproofs}.

\subsubsection*{Example 5}

If \textit{a} and \textit{b} are real numbers, we define $min\{a, b\}$
\begin{align*}
    min\{a, b\} &= a \text{ (if $a < b$)}\\
    min\{a, b\} &= a \text{ (if $a = b$)}\\
    min\{a, b\} &= b \text{ (if $a > b$)}.
\end{align*}

Give a direct proof of the following statement.
\begin{center}
    For all real numbers $d, d_1, d_2, x$, if $d = min\{d_1, d_2\}$ and $x \leq d$, then $x \leq d_1$ and $x \leq d_2$.
\end{center}

We begin by first writing out the hypothesis and conclusion:

\begin{table}[h]
\centering
\begin{tabular}{l}
$d = min\{d_1, d_2\}$ and $x \leq d$ (Hypothesis)\\
$\dots$\\
$x \leq d_1$ and $x \leq d_2$ (Conclusion)
\end{tabular}
\end{table}

To help us better understand, we give an example using real numbers.  Remember, since we are trying to prove a universally quantified statement, one example where the statement is true is not a proof.\\

Let
\begin{align*}
    d_1 &= 2\\
    d_2 &= 4\\
    d &= min\{d_1, d_2\} = 2\\
    x &= 1
\end{align*}

From the definition of $min\{a, b\}$ from above, the minimum $d$ of two numbers, $d_1$ and $d_2$, is equal to one of the two numbers and less than or equal to the other one.  In other words,
\begin{center}
    $d \leq d_1$ and $d \leq d_2$.  
\end{center}

We know that $x \leq d$, so, by the Theorem in Example 2,

\begin{center}
    If $x \leq d$ and $d \leq d_1$, then $x \leq d_1$.\\
    If $x \leq d$ and $d \leq d_2$, then $x \leq d_2$.
\end{center}

The outline of our proof is now:

\begin{table}[h]
\centering
\begin{tabular}{l}
$d = min\{d_1, d_2\}$ and $x \leq d$ (Hypothesis)\\
$d \leq d_1$ and $d \leq d_2$ (Definition of ``minimum'')\\
From $x \leq d$ and $d \leq d_1$, deduce $x \leq d_1$ (Theorem).\\
From $x \leq d$ and $d \leq d_2$, deduce $x \leq d_2$ (Theorem).\\
$x \leq d_1$ and $x \leq d_2$ (Conclusion)
\end{tabular}
\end{table}

\subsection*{Disproving a Universally Quantified Statement}
\addcontentsline{toc}{subsection}{Disproving a Universally Quantified Statement}

Recall that to disprove $\forall x P(x)$, you need to find one $x$ in the domain of discourse such that $P(x)$ is false.  Such a value for x is called a \textit{counterexample}.

\subsection*{Example 6}

The statement
\begin{center}
    $\forall n \in \textbf{Z}^+ (2^n + 1 \text{ is prime})$
\end{center}

is false.  If we assume that it is true, to disprove it, we just need to find one $n$ that is a positive integer where $2^n + 1$ is false.  A counterexample is $n = 3$, because $2^3 + 1 = 9$ is false.

\subsection*{Problem-Solving Tips}
\addcontentsline{toc}{subsection}{Problem-Solving Tips}

To construct a proof of a universally quantified statement, first write down the hypothesis and the conclusion.  To construct the argument, remind yourself what you know about the terms (e.g., ``even,'' ``odd''), symbols (e.g., $X \intersection Y$, $min\{d_1, d_2\}$).  To understand what is to be proved, look at some specific values in the domain.  When proving a universally quantified statement, simply showing the statement is true for specific values is not sufficient; however, it may help you understand the statement.  To disprove a universally quantified statement, find one element $x$ in the domain that makes the propositional function $P(x)$ false.  If asked to prove or disprove a universally quantified statement, you can begin by trying to prove it.  If you succeed, you are finished.  If not, try to disprove it with a counterexample.

\clearpage